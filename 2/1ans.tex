\paragraph{A.}
URL: A Uniform Resource Locator (URL), commonly informally termed a web address (a term which is not defined identically) is a reference to a web resource that specifies its location on a computer network and a mechanism for retrieving it. 

\qquad HTML: Hypertext Markup Language (HTML) is the standard markup language for creating web pages and web applications.

\qquad RTT: In telecommunications, the round-trip delay time (RTD) or round-trip time (RTT) is the length of time it takes for a signal to be sent plus the length of time it takes for an acknowledgment of that signal to be received.

\qquad MIME: Multipurpose Internet Mail Extensions (MIME) is an Internet standard that extends the format of email to support: Text in character sets other than ASCII, Non-text attachments: audio, video, images, application programs etc. Message bodies with multiple parts, Header information in non-ASCII character sets.

\qquad TFTP: Trivial File Transfer Protocol (TFTP) is a simple lockstep File Transfer Protocol which allows a client to get a file from or put a file onto a remote host.

\qquad NFS: Network File System (NFS) is a distributed file system protocol originally developed by Sun Microsystems in 1984, allowing a user on a client computer to access files over a computer network much like local storage is accessed. 

\qquad SNMP: Simple Network Management Protocol (SNMP) is an Internet-standard protocol for collecting and organizing information about managed devices on IP networks and for modifying that information to change device behavior. 

\qquad JPEG: JPEG is a commonly used method of lossy compression for digital images, particularly for those images produced by digital photography. 

\qquad MPEG: The Moving Picture Experts Group (MPEG) is a working group of authorities that was formed by ISO and IEC to set standards for audio and video compression and transmission.