\paragraph{A.}
UDP:
The User Datagram Protocol (UDP) is one of the core members of the Internet protocol suite.

\subitem FSM:
A finite-state machine (FSM) or finite-state automaton (FSA, plural: automata), finite automaton, or simply a state machine, is a mathematical model of computation. It is an abstract machine that can be in exactly one of a finite number of states at any given time. 

\subitem ABR:
Available bit rate (ABR) is a service used in ATM networks when source and destination don't need to be synchronized. ABR does not guarantee against delay or data loss. ABR mechanisms allow the network to allocate the available bandwidth fairly over the present ABR sources. 

\subitem EFCI:
The explicit forward congestion indication bit is a bit that indicates whether the switches are congested at the moment.

\subitem AIMD:
The additive-increase/multiplicative-decrease (AIMD) algorithm is a feedback control algorithm best known for its use in TCP congestion control. AIMD combines linear growth of the congestion window with an exponential reduction when a congestion takes place. 

\subitem MSS:
The maximum segment size (MSS) is a parameter of the options field of the TCP header that specifies the largest amount of data, specified in bytes, that a computer or communications device can receive in a single TCP segment. It does not count the TCP header or the IP header (unlike, for example, the MTU for IP datagrams).